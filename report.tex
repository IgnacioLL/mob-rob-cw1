\documentclass[11pt]{article}
\usepackage{amsmath}
\usepackage{graphicx}
\usepackage{hyperref}
\usepackage{geometry}

\geometry{a4paper, margin=1in}

\title{Particle Filter Implementation for Localization}
\author{Your Name}
\date{\today}

\begin{document}

\maketitle


\section{Introduction}
Localization is a fundamental problem in mobile robotics. This project implements a Particle Filter-based localization algorithm to achieve accurate localization and evaluates its performance against the Adaptive Monte Carlo Localization (AMCL) available in ROS 2.

\section{Algorithm Design}

\subsection{Particle Filter Overview}
The Particle Filter algorithm uses a set of particles to represent the probability distribution of the robot's location. Each particle is evaluated through the given function \texttt{self.sensor\_model.get\_weight} this will return a rational value which a score of how likely is the particle the real position. 

\subsubsection{Motion Update}
In this case the model was uncalibrated as each movement alters too much the expected position of the robot which makes the algorithm unusable when moving. Thus the \texttt{distance\_travelled} variable was divided by 1000. 

\subsubsection{Initial Sampling}
To sample the initial sample of the particle filter algorithm we used a mixture of t-student distribution and normal distribution with $\sigma$ equal to 0.5. 
This was selected instead of a normal distribution because of wanting heavier tails which creats much greater variance. Moreover we used 1000 particles that was the maximum before ros2  overran 

\subsubsection{Resampling}
After each update cycle, particles are resampled to ensure that those with higher weights (i.e., better matches to the observed data) are more likely to survive, while others are discarded. As the difference between weights is actually small, maximum values around 8 and low values around 4 this is accuantated by using the exponent of this weight, after that normalization is done with the sum of every exponentialed weight. 

\subsection{Single Pose Estimate}
To estimate a single pose from the particle set at each timestep, we use the clustering algorithm called DBSCAN. DBSCAN creates multiple clusters using the resampled data points. We pick the biggest cluster and we compute the mean of the points inside the cluster for the position and orientation.

\subsection{Handling the Kidnapped Robot Problem}
The kidnapped robot problem is addressed by using the mixture of normal distribution and t-student distribution which creates random points very far away from the initial pose.

\section{Experimental Analysis}

\subsection{Setup}
The experiments were conducted on the simulated environment using the \texttt{socspioneer} package and the provided map. The localization accuracy and robustness of the Particle Filter were compared to the AMCL implementation in ROS 2.

\subsection{Results}
\begin{itemize}
    \item Localization Accuracy: Describe and compare the accuracy of both implementations.
    \item Robustness to Kidnapping: Discuss how well each implementation handled the kidnapped robot scenario.
    \item Computational Efficiency: Summarize any differences in computational cost between the Particle Filter and AMCL.
\end{itemize}


\section{Conclusion}
In this report, we have implemented and analyzed a Particle Filter for localization, comparing its performance with AMCL. The Particle Filter algorithm demonstrates flexibility in handling dynamic environments and localization challenges such as the kidnapped robot problem. Future work could explore hybrid methods that combine the Particle Filter with Kalman filtering for enhanced localization.

\section*{References}
\begin{itemize}
    \item Thrun, S., Burgard, W., \& Fox, D. (2005). \textit{Probabilistic Robotics}. MIT Press.
    \item ROS Documentation: \url{https://docs.ros.org/}
\end{itemize}

\end{document}
